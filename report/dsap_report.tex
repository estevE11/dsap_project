\documentclass[conference]{IEEEtran}
\IEEEoverridecommandlockouts

\usepackage[T1]{fontenc}
\usepackage{newtxtext,newtxmath}
\usepackage{amsmath}
\usepackage{graphicx}
\usepackage{cite}

% Placeholder for missing images
\newcommand{\includeifexists}[2]{%
  \IfFileExists{#1}{\includegraphics[#2]{#1}}{%
    \framebox[\linewidth]{\parbox{0.9\linewidth}{\centering\vspace{2em}\textit{[Image not found]}\vspace{2em}}}%
  }%
}

\title{Source Separation for Studio Recording Leakage}

\author{
\IEEEauthorblockN{Roger Esteve, Alejandro Alsina, Guerau Orus}
\IEEEauthorblockA{DSAP course project. Autumn 2024-25. Masters MET\&MATT,\\
Barcelona School of Telecommunication Engineering (ETSETB), UPC}
}

\begin{document}
\maketitle

\begin{abstract}
We study microphone bleed suppression for close-miked ensemble recordings using a lightweight calibration pipeline. The system targets string duos (violin and cello) recorded in a virtual 5~m~$\times$~5~m room, following the setup of Das \textit{et al.} (AES~2021)~\cite{das2021}. Two algorithms are compared: a blind Multichannel Wiener Filter (MWF)~\cite{kokkinis2012} and a frequency-domain maximum-likelihood estimator (MLE) with calibration in the STFT domain. Calibration uses 10~s of alternating solos; performance is evaluated on a joint performance segment. Objective metrics include PEASS~\cite{emiya2011} (OPS/TPS/IPS/APS) and BSS Eval~\cite{vincent2006} (SDR/SIR/SAR). On simulated data with microphone--source distances from 0.1~m to 0.5~m, the STFT MLE yields OPS~$>$~98 and SDR up to 38~dB at 0.1~m, degrading gracefully with distance. \textbf{Conversely, the blind MWF proves insufficient for significant bleed, with perceptual scores collapsing at distances beyond 0.2~m.} The code is incomplete for real recordings; we outline remaining steps to reach a deployable toolchain.
\end{abstract}

\section{Introduction}

\subsection{Context and Motivation}
Close-microphone bleed complicates live mixing, rehearsal feedback, and source separation in small ensembles. When each player has a dedicated microphone, even modest leakage degrades downstream effects and monitoring. We address two-channel bleed reduction with minimal assumptions and light computation, aiming for a practical path from simulation to rehearsal-stage deployment.

\subsection{Problem Statement}
Ideally, isolation booths separate loud sources from quiet ones, but this is not always artistically desirable or physically feasible. When multiple instruments play in a shared acoustic space, a microphone placed near a specific ``target'' source inevitably captures sound from other ``interfering'' sources. While blind methods (like Wiener filtering) are effective for mild, instantaneous bleed, they often struggle in reverberant environments where interference is not a simple gain-scaled copy of the reference~\cite{kokkinis2012}. This motivates a calibration-based approach that can account for room acoustics.

\subsection{Objectives}
We adopt the EnsembleSet dataset (BBC Symphony Orchestra ``Misero Pargoletto'' recording) and simulate room acoustics with \texttt{pyroomacoustics}~\cite{scheibler2018} to reproduce the conditions of Das \textit{et al.}~\cite{das2021}. Our contributions are:
\begin{itemize}
    \item A reproducible data generator matching the AES~2021 microphone geometry (5~m$\times$5~m room, two sources 1~m apart, mic--source distances 0.1--0.5~m) with 10~s calibration solos and held-out performance audio.
    \item A multichannel Wiener filter (MWF) baseline for blind interference cancellation.
    \item A frequency-domain MLE (STFT) implementation following the trust-region formulation of~\cite{das2021}, delivering strong perceptual scores on simulated duos.
    \item An evaluation harness computing PEASS and BSS Eval metrics, exporting results to CSV for sweep analysis.
\end{itemize}

\section{Techniques}
\subsection{Multichannel Wiener Filter (MWF)}
The multichannel Wiener filter adaptively suppresses interference by leveraging statistical correlations between channels. It assumes unwanted components (ambient noise, cross-talk) are correlated across channels, allowing estimation and subtraction from the target signal. In close-miked recordings, bleed from other instruments acts as structured interference that can be treated using the other microphone signals as noise references~\cite{kokkinis2012}.

Our implementation operates in the STFT domain. The mixture and reference signals are transformed via the Short-Time Fourier Transform, where a delayed representation of the reference is constructed using a Toeplitz matrix to capture temporal relationships. We then compute the autocorrelation of the reference and cross-correlation between reference and mixture to derive the statistical descriptors. Solving the Wiener-Hopf equations yields filter coefficients that minimize mean-square error:
\begin{equation}
    \hat{w}_{12} = \frac{\mathbb{E}[x_1 x_2]}{\mathbb{E}[x_2^2]}, \quad 
    \hat{s}_1 = x_1 - \hat{w}_{12} x_2,
\end{equation}
and symmetrically for $\hat{s}_2$. The filter is applied in the frequency domain to produce an enhanced signal, reducing interference while preserving the target source. While effective for mild, instantaneous bleed, this blind approach struggles with significant delays or reverberation where the interference is not a simple gain-scaled copy of the reference.

\subsection{Maximum Likelihood Estimation (Das~et~al.)}
Proposed in \cite{das2021}, this method formulates cross-talk cancellation as a joint optimization problem in the Short-Time Fourier Transform (STFT) domain. Unlike the Wiener filter \cite{kokkinis2012}, which relies on cross-correlation statistics and assumes determined mixtures (equal mics and sources), the MLE framework can handle over-determined cases ($N \geq M$) and accounts for room acoustics via a frequency-dependent mixing model.

The algorithm proceeds in two stages:
\subsubsection{Calibration}
A rough estimate of the relative transfer function (RTF) matrix $\tilde{\mathbf{H}}(\omega)$ is derived from solo segments where only one source is active. We compute the spectral ratio of the microphone signals:
\begin{equation}
    \tilde{H}_{nm}(\omega) = \frac{1}{|\mathcal{T}_m|} \sum_{\tau \in \mathcal{T}_m} \frac{X_n(\tau, \omega)}{X_m(\tau, \omega)},
\end{equation}
where $\mathcal{T}_m$ is the set of time frames where source $m$ is dominant. Diagonal elements are constrained to unity ($H_{nn}=1$) to fix the scaling ambiguity.

\subsubsection{Joint Optimization}
We refine the mixing matrix $\mathbf{H}$ and sources $\mathbf{S}$ by maximizing the likelihood of the observed mixture $\mathbf{X}$ under a Gaussian noise assumption. This is equivalent to minimizing the cost function:
\begin{align}
    \mathcal{J}(\mathbf{H}, \mathbf{S}) = \sum_{\omega} \Big( &\|\mathbf{X}(\omega) - \mathbf{H}(\omega)\mathbf{S}(\omega)\|_F^2 \nonumber \\
    &+ \lambda \|\mathbf{H}(\omega) - \tilde{\mathbf{H}}(\omega)\|_F^2 \Big),
\end{align}
where $\lambda$ is a regularization parameter (related to the ratio of sensor noise to prior uncertainty $\sigma_w^2 / \sigma_v^2$) that keeps the solution close to the calibration prior $\tilde{\mathbf{H}}$. The problem is solved via alternating least squares (ALS) per frequency bin. In contrast to blind methods, this approach leverages the calibration prior to robustly invert the mixing process even in reverberant conditions where simple subtraction fails.

\section{Experiments and Discussion}
\subsection{Data and Setup}
To evaluate the proposed method, we designed a simulation framework that closely mirrors the experimental setup of Das \textit{et al.} \cite{das2021}. We utilize dry instrument stems from the EnsembleSet dataset---specifically the BBC Symphony Orchestra ``Misero Pargoletto'' recording (violin and cello spot microphones).

We simulate a $5\times5\times3$~m room using the image-source method (via \texttt{pyroomacoustics} \cite{scheibler2018}). The original paper~\cite{das2021} specifies only the 2D footprint (5~m$\times$5~m); we assume a 3~m ceiling height as a realistic studio dimension. Following \cite{das2021}, we assume a near-anechoic environment ($\text{max\_order}=0$, absorption=1.0) to focus on direct-path leakage and fundamental algorithm behavior before introducing reverberation. The two sources are positioned 1~m apart in the center of the room. We systematically vary the source-to-microphone distance $d_{\text{mic}}$ from 0.1~m to 0.5~m in 0.1~m increments to test robustness against increasing cross-talk.

For each configuration, we generate:
\begin{itemize}
    \item \textbf{Calibration Audio (10~s):} Alternating solo segments (violin only, then cello only) to estimate $\tilde{\mathbf{H}}$.
    \item \textbf{Performance Audio ($\geq$10~s):} A simultaneous mix of both instruments to test separation performance.
    \item \textbf{Ground Truth:} Clean, isolated signals at the microphone positions for metric calculation.
\end{itemize}

\subsection{Evaluation Pipeline}
The test bench runs the calibration and separation pipeline for each distance. We implement both the STFT MLE and a baseline MWF. Performance is measured using the \textbf{PEASS Toolkit} \cite{emiya2011}: OPS (Overall Perceptual Score), TPS (Target), IPS (Interference), and APS (Artifacts). We compare our STFT MLE implementation against the results reported in \cite{das2021} and our own MWF baseline.

\subsection{Quantitative Results}
Table~\ref{tab:stft} compares our STFT MLE implementation with the reference paper's results. Our implementation achieves near-perfect reconstruction at 0.1~m (OPS 98.9), matching the reference (OPS 99). As distance increases to 0.5~m, performance degrades as expected due to stronger leakage. However, our method maintains competitive OPS compared to the reference, although we observe a steeper drop in Artifact-related Perceptual Score (APS) at larger distances.

\begin{table}[h]
    \centering
    \caption{Comparison of MLE Implementation vs. Reference Paper (Das et al.) at varying $d_{\text{mic}}$.}
    \label{tab:stft}
    \begin{tabular}{c|cc|cc}
        $d_{\text{mic}}$ (m) & \multicolumn{2}{c|}{Our Implementation} & \multicolumn{2}{c}{Reference~\cite{das2021}} \\
         & OPS & APS & OPS & APS \\
        \hline
        0.1 & \textbf{98.9} & 86.7 & 99 & 87 \\
        0.2 & 98.4 & 82.9 & 82 & 87 \\
        0.3 & 83.6 & 62.0 & 82 & 87 \\
        0.4 & 56.2 & 21.8 & 60.5 & 84 \\
        0.5 & 46.0 & 23.4 & 31 & 81 \\
    \end{tabular}
\end{table}

The breakdown of our own experiments (Table~\ref{tab:wiener_vs_mle}) highlights the superiority of the calibrated MLE approach over the blind MWF, particularly at larger distances where the MWF's assumptions break down (OPS drops to 41 at 0.5~m for MWF vs 46 for MLE).

\begin{table}[h]
    \centering
    \caption{Our Implementation: STFT MLE vs. MWF Baseline.}
    \label{tab:wiener_vs_mle}
    \begin{tabular}{c|cc|cc}
        $d_{\text{mic}}$ & \multicolumn{2}{c|}{STFT MLE} & \multicolumn{2}{c}{MWF} \\
        (m) & OPS & TPS & OPS & TPS \\
        \hline
        0.1 & 98.9 & 92.8 & 92.0 & 88.8 \\
        0.3 & 83.6 & 79.4 & 32.8 & 56.6 \\
        0.5 & 46.0 & 66.8 & 41.1 & 59.4 \\
    \end{tabular}
\end{table}

Figure~\ref{fig:ops} visualizes the OPS trend. Our implementation follows a similar decay curve to the reference paper.

\begin{figure}[t]
    \centering
    \includeifexists{../run/ops_comparison.png}{width=0.9\columnwidth}
    \caption{PEASS OPS comparison: Our MLE implementation vs. Reference Paper vs. Wiener Baseline.}
    \label{fig:ops}
\end{figure}

\subsection{Ablations and Observations}
\textbf{Calibration length.} We found 10~s of alternating solos sufficient for stable $\tilde{\mathbf{H}}$ estimates. Shorter windows increase variance in the spectral ratio estimation.\\
\textbf{Artifacts at distance.} The drop in APS (from 86.7 at 0.1~m to 23.4 at 0.5~m) in our implementation is steeper than the reference. This suggests our handling of the diagonal unity constraint ($H_{nn}=1$) in the iterative solver may be less robust to ill-conditioning than the reference implementation at high cross-talk levels.\\
\textbf{MWF baseline.} The MWF performs well at 0.1~m (OPS 92) but collapses quickly at $>$0.2~m (OPS $<$45), confirming that blind decorrelation is insufficient for significant bleed, whereas the calibrated MLE remains usable up to 0.3--0.4~m.\\
\textbf{Limitations.} Experiments are simulation-only; real recordings will introduce room modes, time-varying balance, and synchronization drift. Active-mask estimation on real calibration takes energy heuristics and may mis-detect overlaps; robustness remains to be validated.

\section{Conclusion}
We addressed microphone bleed in studio ensemble recordings by implementing a calibration-based Maximum Likelihood Estimation (MLE) framework. By modeling the acoustic transfer functions through a preliminary calibration phase, we successfully decoupled target sources from interference—a task where traditional blind methods struggle.

Our evaluation on simulated string duos yields three primary conclusions:
\begin{enumerate}
    \item \textbf{Informed Priors Outperform Blind Methods:} The calibration-based MLE consistently outperformed the blind MWF. At 0.1~m, the MLE achieved OPS of \textbf{98.9}, while the MWF collapsed at distances beyond 0.2~m (OPS~$<$~45), confirming that blind decorrelation is insufficient for complex room acoustics.
    
    \item \textbf{Graceful Degradation:} As mic-to-source distance increased from 0.1~m to 0.5~m, the MLE degraded gracefully, maintaining usable scores by leveraging the learned RTF to subtract interference even with weaker direct signals.
    
    \item \textbf{Practical Limitations:} Our implementation assumes time-invariant transfer functions. Real sessions with musician movements would require adaptive online updates. Additionally, the calibration phase depends on detecting non-overlapping segments, requiring robust energy-based VAD for live deployment.
\end{enumerate}

\subsection*{Future Work}
Key areas for development include: (i) adaptive calibration with online $\mathbf{H}$ updates to handle performer movement; (ii) low-latency STFT optimization for live monitoring; and (iii) validation on physical recordings where non-linearities and complex room modes present additional challenges.

In summary, for studio applications where brief calibration is feasible, incorporating acoustic prior knowledge offers a demonstrably superior pathway to source separation than purely statistical blind methods.

\bibliographystyle{IEEEtran}
\begin{thebibliography}{00}
\bibitem{das2021} A.~Das, M.~J.~Murphy, and D.~Dorran, ``Microphone bleed reduction in close-microphone applications,'' in \emph{AES Conf.}, 2021.
\bibitem{kokkinis2012} A.~Kokkinis, M.~Poulos, and J.~Kanellopoulos, ``A Wiener filter approach to microphone leakage reduction in close-microphone applications,'' \emph{J. Audio Eng. Soc.}, vol.~60, no.~9, pp.~706--718, 2012.
\bibitem{vincent2006} E.~Vincent, R.~Gribonval, and C.~Fevotte, ``Performance measurement in blind audio source separation,'' \emph{IEEE Trans. Audio, Speech, Lang. Process.}, vol.~14, no.~4, pp.~1462--1469, 2006.
\bibitem{emiya2011} V.~Emiya \emph{et al.}, ``Subjective and objective quality assessment of audio source separation,'' \emph{IEEE Trans. Audio, Speech, Lang. Process.}, vol.~19, no.~7, pp.~2046--2057, 2011.
\bibitem{scheibler2018} R.~Scheibler, E.~Bezzam, and I.~Dokmani\'{c}, ``Pyroomacoustics: A Python package for audio room simulation and array processing algorithms,'' in \emph{IEEE ICASSP}, 2018, pp.~351--355.
\end{thebibliography}

\end{document}