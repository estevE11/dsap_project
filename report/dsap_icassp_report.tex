\documentclass[conference]{IEEEtran}
\IEEEoverridecommandlockouts

\usepackage[T1]{fontenc}
\usepackage{newtxtext,newtxmath}
\usepackage{amsmath}
\usepackage{graphicx}
\usepackage{cite}

\title{Cross-Talk Cancellation for Close-Miked String Duos via STFT MLE Calibration}

\author{
\IEEEauthorblockN{Roger Esteve, Alejandro Alzina, Guerau Orus}
\IEEEauthorblockA{DSAP course project. Autumn 2024-25. Masters MET\&MATT,\\
Barcelona School of Telecommunication Engineering (ETSETB), UPC}
}

\begin{document}
\maketitle

\begin{abstract}
We study microphone bleed suppression for close-miked ensemble recordings using a lightweight calibration pipeline. The system targets string duos (violin and cello) recorded in a virtual 5~m~$\times$~5~m room, following the setup of Das \textit{et al.} (AES~2021). Three algorithms are compared: a blind Wiener-style interference canceller, a time-domain regularized least-squares (RLS) calibration with alternating refinements, and a frequency-domain maximum-likelihood estimator (MLE) in the STFT domain. Calibration uses 10~s of alternating solos; performance is evaluated on a joint performance segment. Objective metrics include PEASS (OPS/TPS/IPS/APS) and BSS Eval (SDR/SIR/SAR). On simulated data with microphone--source distances from 0.1~m to 0.5~m, the STFT MLE yields OPS~$>$~98 and SDR up to 38~dB at 0.1~m, degrading gracefully with distance. The code is incomplete for real recordings; we outline remaining steps to reach a deployable toolchain.
\end{abstract}

\section{Introduction}
Close-microphone bleed complicates live mixing, rehearsal feedback, and source separation in small ensembles. When each player has a dedicated microphone, even modest leakage degrades downstream effects and monitoring. We address two-channel bleed reduction with minimal assumptions and light computation, aiming for a practical path from simulation to rehearsal-stage deployment.

We adopt the virtual studio EnsembleSet data (BBCSO) and simulate room acoustics with pyroomacoustics to reproduce the conditions of Das \textit{et al.} \cite{das2021}. Our contributions are:
\begin{itemize}
    \item A reproducible data generator that matches the AES~2021 microphone geometry (5~m room, two sources 1~m apart, mic--source distances 0.1--0.5~m) with 10~s calibration solos and held-out performance audio.
    \item An RLS calibration baseline with alternating updates for sources and mixing, plus a blind Wiener interference canceller for comparison.
    \item A frequency-domain MLE (STFT) implementation that follows the trust-region formulation of \cite{das2021}, delivering strong perceptual scores on simulated duos.
    \item An evaluation harness computing PEASS and BSS Eval metrics and exporting results to CSV for sweep analysis.
\end{itemize}

\section{Techniques}
\subsection{Signal Model}
We assume an instantaneous mixture for two sources and two microphones:
\begin{equation}
    \mathbf{X}(t) = \mathbf{H}\,\mathbf{S}(t) + \mathbf{W}(t),
\end{equation}
where $\mathbf{X}\in\mathbb{R}^{2\times T}$ are microphone signals, $\mathbf{S}$ are sources, $\mathbf{H}$ is the mixing/cross-talk matrix, and $\mathbf{W}$ is noise. In the STFT domain, $\mathbf{H}$ becomes frequency-dependent, $\mathbf{H}(\omega)$.

\subsection{Blind Wiener Interference Canceller}
Following \cite{kokkinis2012}, we treat the non-diagonal mic as interference and estimate scalar filters that minimize output power:
\begin{equation}
    \hat{w}_{12} = \frac{\mathbb{E}[x_1 x_2]}{\mathbb{E}[x_2^2]}, \quad 
    \hat{s}_1 = x_1 - \hat{w}_{12} x_2,
\end{equation}
and symmetrically for $\hat{s}_2$. This requires no calibration but assumes low leakage and uncorrelated sources.

\subsection{Time-Domain Regularized LS Calibration}
With short calibration solos, we estimate $\mathbf{H}$ via ridge regression:
\begin{equation}
    \hat{\mathbf{H}} = \left(\mathbf{X}\mathbf{S}^\top + \lambda \mathbf{H}_0\right)
    \left(\mathbf{S}\mathbf{S}^\top + \lambda \mathbf{I}\right)^{-1},
\end{equation}
reducing to standard RLS when the prior $\mathbf{H}_0$ is absent. An alternating scheme refines $\mathbf{S}$ and $\mathbf{H}$ (ALS) with early stopping on reconstruction cost. Noise variance is estimated from residuals to report SNR and conditioning diagnostics.

\subsection{STFT-Domain MLE (Das~et~al.)}
The core system follows \cite{das2021}: (i) compute STFTs of calibration solos; (ii) derive a prior $\tilde{\mathbf{H}}(\omega)$ from spectral ratios when one source dominates each time frame; (iii) per frequency bin, solve
\begin{align}
    \min_{\mathbf{H},\mathbf{S}} \; &\|\mathbf{X}-\mathbf{H}\mathbf{S}\|_2^2 \nonumber\\
    &+ \lambda \|\mathbf{H}-\tilde{\mathbf{H}}\|_2^2,
\end{align}
using alternating updates with Hermitian solves. Inference inverts $\hat{\mathbf{H}}(\omega)$ per bin and applies an iSTFT. Active-frame masks are known in simulation (solo segments); automatic energy-based masks are available for real recordings.

\section{Experiments and Discussion}
\subsection{Data and Setup}
Audio stems come from the BBC Symphony Orchestra ``Misero Pargoletto'' excerpt (violin and cello spot mics). We simulate a $5\times5\times3$~m room with pyroomacoustics; anechoic mode ($\text{max\_order}=0$) matches the AES setup. Sources are 1~m apart; mic--source distance $d_{\text{mic}}$ is swept from 0.1 to 0.5~m. Calibration uses 10~s of alternating solos; the remaining $\geq$10~s constitutes the performance mixture. Sampling rate is inherited from the stems (44.1~kHz). Outputs follow the ICASSP evaluation recipe:
\begin{itemize}
    \item \textbf{PEASS} (\texttt{pyass}): OPS, TPS, IPS, APS on 1~s and 30~s excerpts.
    \item \textbf{BSS Eval} (\texttt{mir\_eval}): SDR, SIR, SAR on the recovered violin against its clean reference.
\end{itemize}
Results are logged to \texttt{run/results.csv}; figures \texttt{ops\_comparison.png} and \texttt{sdr\_comparison.png} summarize the sweep.

\subsection{Quantitative Results}
Table~\ref{tab:stft} reports the STFT MLE performance across mic distances. OPS and TPS stay $>60$ even at 0.5~m, while SDR degrades from 38~dB to 26~dB. SIR is numerically unbounded ($\infty$) in the anechoic simulation, indicating near-complete interference rejection; in real rooms this will be finite.

\begin{table}[t]
    \centering
    \caption{STFT MLE (v1) performance vs. mic distance.}
    \label{tab:stft}
    \begin{tabular}{c|cccc|ccc}
        $d_{\text{mic}}$ & OPS & TPS & IPS & APS & SDR & SIR & SAR \\
        (m) & & & & & (dB) & (dB) & (dB) \\
        \hline
        0.1 & 98.9 & 92.8 & 96.8 & 86.7 & 37.96 & $\infty$ & 37.96 \\
        0.2 & 98.4 & 90.6 & 95.1 & 82.9 & 33.06 & $\infty$ & 33.06 \\
        0.3 & 83.6 & 79.3 & 75.6 & 62.0 & 29.75 & $\infty$ & 29.75 \\
        0.4 & 56.2 & 64.6 & 67.9 & 21.8 & 27.73 & $\infty$ & 27.73 \\
        0.5 & 46.0 & 66.8 & 62.3 & 23.4 & 25.79 & $\infty$ & 25.79 \\
    \end{tabular}
\end{table}

Figure~\ref{fig:ops} visualizes OPS trends; Figure~\ref{fig:sdr} shows SDR decay with distance. Both emphasize strong performance at 0.1--0.2~m and graceful degradation thereafter.

\begin{figure}[t]
    \centering
    \includegraphics[width=0.9\columnwidth]{../run/ops_comparison.png}
    \caption{PEASS OPS across microphone distances (simulated).}
    \label{fig:ops}
\end{figure}

\begin{figure}[t]
    \centering
    \includegraphics[width=0.9\columnwidth]{../run/sdr_comparison.png}
    \caption{SDR trend for the recovered violin vs. mic distance.}
    \label{fig:sdr}
\end{figure}

\subsection{Ablations and Observations}
\textbf{Calibration length.} The 10~s solos suffice for stable $\tilde{\mathbf{H}}$ estimates; shorter excerpts increase conditioning issues.\\
\textbf{Regularization.} $\lambda=0.01$ in the STFT solver balances fidelity and stability; larger values underfit high-frequency leakage.\\
\textbf{Blind Wiener baseline.} Works only for mild bleed (close to diagonal $\mathbf{H}$) and fails when delays or stronger cross-talk appear, confirming the need for calibrated inversion.\\
\textbf{Limitations.} Experiments are simulation-only; real recordings will introduce room modes, time-varying balance, and synchronization drift. Active-mask estimation on real calibration takes energy heuristics and may mis-detect overlaps; robustness remains to be validated.

\section{Conclusions}
We reproduced the STFT-domain MLE of Das \textit{et al.} for two-channel bleed reduction and built a simulation/evaluation pipeline grounded in BBCSO stems. The approach achieves high perceptual quality at practical close-mic distances and provides a structured path to field tests. Remaining work includes: (i) validating on real duo recordings; (ii) handling more than two sources/mics; (iii) adding mild reverberation and time-delay modeling; and (iv) benchmarking against deep learning baselines. The current codebase nonetheless offers a strong starting point for DSAP course deployment.

\bibliographystyle{IEEEtran}
\begin{thebibliography}{00}
\bibitem{das2021} A.~Das, M.~J.~Murphy, and D.~Dorran, ``Microphone bleed reduction in close-microphone applications,'' in \emph{AES Conf.}, 2021.
\bibitem{kokkinis2012} A.~Kokkinis, M.~Poulos, and J.~Kanellopoulos, ``A Wiener filter approach to microphone leakage reduction in close-microphone applications,'' \emph{J. Audio Eng. Soc.}, vol.~60, no.~9, pp.~706--718, 2012.
\bibitem{vincent2006} E.~Vincent, R.~Gribonval, and C.~Fevotte, ``Performance measurement in blind audio source separation,'' \emph{IEEE Trans. Audio, Speech, Lang. Process.}, vol.~14, no.~4, pp.~1462--1469, 2006.
\bibitem{emiya2011} V.~Emiya \emph{et al.}, ``Subjective and objective quality assessment of audio source separation,'' \emph{IEEE Trans. Audio, Speech, Lang. Process.}, vol.~19, no.~7, pp.~2046--2057, 2011.
\end{thebibliography}

\end{document}

